% $Log: abstract.tex,v $
% Revision 1.1  93/05/14  14:56:25  starflt
% Initial revision
% 
% Revision 1.1  90/05/04  10:41:01  lwvanels
% Initial revision
% 
%
%% The text of your abstract and nothing else (other than comments) goes here.
%% It will be single-spaced and the rest of the text that is supposed to go on
%% the abstract page will be generated by the abstractpage environment.  This
%% file should be \input (not \include 'd) from cover.tex.

  This thesis explores how fork-join parallelism, as supported by
  concurrency platforms such as Cilk and OpenMP, can be embedded into
  a compiler's intermediate representation~(IR).  Mainstream compilers
  typically treat parallel linguistic constructs as syntactic sugar
  for function calls into a parallel runtime.  These calls prevent the
  compiler from performing optimizations across parallel control
  constructs.  Remedying this situation is generally thought to
  require an extensive reworking of compiler analyses and code
  transformations to handle parallel semantics.

  \tapir is a compiler IR that represents logically parallel tasks
  asymmetrically in the program's control flow graph.  \tapir allows
  the compiler to optimize across parallel control constructs with
  only minor changes to its existing analyses and code
  transformations.  To prototype \tapir in the LLVM compiler, for
  example, the \tapir team added or modified about $\fillintheblank{6000}$ lines of
  LLVM's $\fillintheblank{4}$-million-line codebase.  \tapir enables
  LLVM's existing compiler optimizations for serial code --- including
  loop-invariant-code motion, common-subexpression elimination, and
  tail-recursion elimination --- to work with parallel control
  constructs such as spawning and parallel loops.  Tapir also supports
  parallel optimizations such as loop scheduling.

  This research reported in this thesis represents joint work with Tao B. Schardl and Charles E. Leiserson.

