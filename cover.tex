% -*-latex-*-
% 
% For questions, comments, concerns or complaints:
% thesis@mit.edu
% 
%
% $Log: cover.tex,v $
% Revision 1.8  2008/05/13 15:02:15  jdreed
% Degree month is June, not May.  Added note about prevdegrees.
% Arthur Smith's title updated
%
% Revision 1.7  2001/02/08 18:53:16  boojum
% changed some \newpages to \cleardoublepages
%
% Revision 1.6  1999/10/21 14:49:31  boojum
% changed comment referring to documentstyle
%
% Revision 1.5  1999/10/21 14:39:04  boojum
% *** empty log message ***
%
% Revision 1.4  1997/04/18  17:54:10  othomas
% added page numbers on abstract and cover, and made 1 abstract
% page the default rather than 2.  (anne hunter tells me this
% is the new institute standard.)
%
% Revision 1.4  1997/04/18  17:54:10  othomas
% added page numbers on abstract and cover, and made 1 abstract
% page the default rather than 2.  (anne hunter tells me this
% is the new institute standard.)
%
% Revision 1.3  93/05/17  17:06:29  starflt
% Added acknowledgements section (suggested by tompalka)
% 
% Revision 1.2  92/04/22  13:13:13  epeisach
% Fixes for 1991 course 6 requirements
% Phrase "and to grant others the right to do so" has been added to 
% permission clause
% Second copy of abstract is not counted as separate pages so numbering works
% out
% 
% Revision 1.1  92/04/22  13:08:20  epeisach

% NOTE:
% These templates make an effort to conform to the MIT Thesis specifications,
% however the specifications can change.  We recommend that you verify the
% layout of your title page with your thesis advisor and/or the MIT 
% Libraries before printing your final copy.
\title{How Should Compilers Represent Fork-Join Parallelism?}

\author{William S. Moses}
% If you wish to list your previous degrees on the cover page, use the 
% previous degrees command:
%       \prevdegrees{A.A., Harvard University (1985)}
% You can use the \\ command to list multiple previous degrees
%       \prevdegrees{B.S., University of California (1978) \\
%                    S.M., Massachusetts Institute of Technology (1981)}
\department{Department of Electrical Engineering and Computer Science}

% If the thesis is for two degrees simultaneously, list them both
% separated by \and like this:
% \degree{Doctor of Philosophy \and Master of Science}
\degree{Master of Engineering in Electrical Engineering and Computer Science}

% As of the 2007-08 academic year, valid degree months are September, 
% February, or June.  The default is June.
\degreemonth{June}
\degreeyear{2017}
\thesisdate{May 25, 2017}

%% By default, the thesis will be copyrighted to MIT.  If you need to copyright
%% the thesis to yourself, just specify the `vi' documentclass option.  If for
%% some reason you want to exactly specify the copyright notice text, you can
%% use the \copyrightnoticetext command.  
%\copyrightnoticetext{\copyright IBM, 1990.  Do not open till Xmas.}

% If there is more than one supervisor, use the \supervisor command
% once for each.
\supervisor{Charles E. Leiserson}{Professor}
\supervisor{Tao B. Schardl}{Postdoctoral Associate}

% This is the department committee chairman, not the thesis committee
% chairman.  You should replace this with your Department's Committee
% Chairman.
\chairman{Christopher Terman}{Chairman, Masters of Engineering Thesis Committee}

% Make the titlepage based on the above information.  If you need
% something special and can't use the standard form, you can specify
% the exact text of the titlepage yourself.  Put it in a titlepage
% environment and leave blank lines where you want vertical space.
% The spaces will be adjusted to fill the entire page.  The dotted
% lines for the signatures are made with the \signature command.
\maketitle

% The abstractpage environment sets up everything on the page except
% the text itself.  The title and other header material are put at the
% top of the page, and the supervisors are listed at the bottom.  A
% new page is begun both before and after.  Of course, an abstract may
% be more than one page itself.  If you need more control over the
% format of the page, you can use the abstract environment, which puts
% the word "Abstract" at the beginning and single spaces its text.

%% You can either \input (*not* \include) your abstract file, or you can put
%% the text of the abstract directly between the \begin{abstractpage} and
%% \end{abstractpage} commands.

% First copy: start a new page, and save the page number.

\cleardoublepage

% Uncomment the next line if you do NOT want a page number on your
% abstract and acknowledgments pages.
% \pagestyle{empty}

\setcounter{savepage}{\thepage}
\begin{abstractpage}
% $Log: abstract.tex,v $
% Revision 1.1  93/05/14  14:56:25  starflt
% Initial revision
% 
% Revision 1.1  90/05/04  10:41:01  lwvanels
% Initial revision
% 
%
%% The text of your abstract and nothing else (other than comments) goes here.
%% It will be single-spaced and the rest of the text that is supposed to go on
%% the abstract page will be generated by the abstractpage environment.  This
%% file should be \input (not \include 'd) from cover.tex.

  This thesis explores how fork-join parallelism, as supported by
  concurrency platforms such as Cilk and OpenMP, can be embedded into
  a compiler's intermediate representation~(IR).  Mainstream compilers
  typically treat parallel linguistic constructs as syntactic sugar
  for function calls into a parallel runtime.  These calls prevent the
  compiler from performing optimizations across parallel control
  constructs.  Remedying this situation is generally thought to
  require an extensive reworking of compiler analyses and code
  transformations to handle parallel semantics.

  \tapir is a compiler IR that represents logically parallel tasks
  asymmetrically in the program's control flow graph.  \tapir allows
  the compiler to optimize across parallel control constructs with
  only minor changes to its existing analyses and code
  transformations.  To prototype \tapir in the LLVM compiler, for
  example, the \tapir team added or modified about $\fillintheblank{6000}$ lines of
  LLVM's $\fillintheblank{4}$-million-line codebase.  \tapir enables
  LLVM's existing compiler optimizations for serial code --- including
  loop-invariant-code motion, common-subexpression elimination, and
  tail-recursion elimination --- to work with parallel control
  constructs such as spawning and parallel loops.  Tapir also supports
  parallel optimizations such as loop scheduling.

  This research reported in this thesis represents joint work with Tao B. Schardl and Charles E. Leiserson.


\end{abstractpage}

% Additional copy: start a new page, and reset the page number.  This way,
% the second copy of the abstract is not counted as separate pages.
% Uncomment the next 6 lines if you need two copies of the abstract
% page.
% \setcounter{page}{\thesavepage}
% \begin{abstractpage}
% % $Log: abstract.tex,v $
% Revision 1.1  93/05/14  14:56:25  starflt
% Initial revision
% 
% Revision 1.1  90/05/04  10:41:01  lwvanels
% Initial revision
% 
%
%% The text of your abstract and nothing else (other than comments) goes here.
%% It will be single-spaced and the rest of the text that is supposed to go on
%% the abstract page will be generated by the abstractpage environment.  This
%% file should be \input (not \include 'd) from cover.tex.

  This thesis explores how fork-join parallelism, as supported by
  concurrency platforms such as Cilk and OpenMP, can be embedded into
  a compiler's intermediate representation~(IR).  Mainstream compilers
  typically treat parallel linguistic constructs as syntactic sugar
  for function calls into a parallel runtime.  These calls prevent the
  compiler from performing optimizations across parallel control
  constructs.  Remedying this situation is generally thought to
  require an extensive reworking of compiler analyses and code
  transformations to handle parallel semantics.

  \tapir is a compiler IR that represents logically parallel tasks
  asymmetrically in the program's control flow graph.  \tapir allows
  the compiler to optimize across parallel control constructs with
  only minor changes to its existing analyses and code
  transformations.  To prototype \tapir in the LLVM compiler, for
  example, the \tapir team added or modified about $\fillintheblank{6000}$ lines of
  LLVM's $\fillintheblank{4}$-million-line codebase.  \tapir enables
  LLVM's existing compiler optimizations for serial code --- including
  loop-invariant-code motion, common-subexpression elimination, and
  tail-recursion elimination --- to work with parallel control
  constructs such as spawning and parallel loops.  Tapir also supports
  parallel optimizations such as loop scheduling.

  This research reported in this thesis represents joint work with Tao B. Schardl and Charles E. Leiserson.


% \end{abstractpage}

\cleardoublepage

\section*{Acknowledgments}
This thesis is derived from a project done in collaboration with Tao B. Schardl and Charles E. Leiserson.
Together, we form the ``Tapir team,'' which I will refer to in this document.
TB and Charles were instrumental in helping talk through many of the ideas that inspired the following work.
Much of this content was previously published at the 2017 Symposium on Principles and Practice of Parallel Programming (PPoPP)\cite{tapir}. I would like to thank the PPoPP reviewers for their helpful comments
and their selection as best paper.

I would also like to thank my professional colleagues.
I would like to thank Tim Kaler and the students of the Fall 2016 MIT class
6.172/6.871 \textit{Performance Evaluation of Software Systems} for
their patience in using the Tapir/LLVM compiler throughout the
semester and reporting bugs. Additional thanks
to Shahin Kamali, Bradley Kuszmaul, Bojan Serafimov, Jiahao Li, Dougie Kogut,
and the entire MIT Supertech research group for many helpful discussions.
Further thanks to Larry Hardesty of the MIT News Office for asking
questions that helping simplify \figref{loc_breakdown}. I would also like to
thank Johannes Doerfert, Simon Moll, Vikram Adve, and
Hal Finkel for a number of discussions on expanding Tapir and trying to bring
its ideas into LLVM. 

Finally, I would like to thank my family without whose constant love and
support I would never have been able to make it to where I am today.
I would like my parents, John and Marina Moses, my sister Sophia Moses,
my grandfather Panayoti Stefanidis, and my dog Patches.

This research was supported in part by NSF Grants 1314547\punt{Parlay}
and 1533644\punt{XPS}, in part by a MIT CSAIL grant from Foxconn,
in part by the Intelligence Advanced Research Projects Activity
(IARPA) via Department of Interior/ Interior Business Center (DoI/IBC)
contract number D16PC00002, and in part by an MIT EECS SuperUROP\@.
The U.S. Government is authorized to
reproduce and distribute reprints for Governmental purposes
notwithstanding any copyright annotation thereon.  Disclaimer: The
views and conclusions contained herein are those of the authors and
should not be interpreted as necessarily representing the official
policies or endorsements, either expressed or implied, of IARPA,
DoI/IBC, or the U.S. Government.


%%%%%%%%%%%%%%%%%%%%%%%%%%%%%%%%%%%%%%%%%%%%%%%%%%%%%%%%%%%%%%%%%%%%%%
% -*-latex-*-
