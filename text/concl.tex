\chapput{concl}{Conclusion}

To conclude, I would like to leave the reader with three interesting
considerations regarding the nature of asymmetry in parallelism, the
future of parallel optimizations, and extensions of \tapir-like
systems to other models of parallel programming.

Reasoning about logically parallel tasks asymmetrically based on
serial semantics can sometimes simplify the understanding of a
parallel program's behavior.  When a task is spawned to execute in
parallel with another, it is natural to reason about the logically
parallel tasks as symmetric, because their instructions can execute in
any relative order.  For parallel programs with serial semantics,
however, it is always valid to execute the program on a single
processor, which asymmetrically executes one parallel task to
completion before starting the other.  Serial semantics encourage an
asymmetric representation of parallel control flow that is similar
enough to its serial elision that most common analyses and
transformations for serial programs work on parallel constructs with
little or no modification.  In particular, serial semantics enables
common optimizations on parallel code that can be invalid under other
models of parallelism~\cite{VafeiadisBaCh15}.

One of the great benefits of \tapir is that its strategy for
representing parallelism makes it easy to write optimization passes
specifically for parallel code.  \chapref{opt} briefly mentioned some
parallel optimization passes we implemented, including parallel-loop
scheduling and unnecessary-sync elimination.  In addition to helping
close the performance gap between serial and parallel versions of
code, we hope that the introduction of \tapir will encourage the
development and implementation of many more parallel-optimization
passes.

Finally, \tapir allows fork-join parallel programs to benefit from
both serial and parallel optimizations.  Moving forwards, it is
natural to wonder whether other models of parallelism, such as
pipeline parallelism \cite{LeeLeSc15, NavarroAsTa09, DuFeAg03} or
data-graph computations \cite{LowGoKy10, LowBiGo12, MalewiczAuBi10,
  NguyenLePi13, NguyenLePi14, ShunBl13, ShunDhBl15}, can take
advantage of the \tapir approach.


% \begin{closeitemize}
% \item The \code{testCilk.sh} script will recompile and rerun all tests
%   from the MIT Cilk benchmark test suite.

% \item The \code{testIntel.sh} script will recompile and rerun all
%   tests from the set of Intel Cilk Plus example programs.

% \item The \code{testPBBS.sh} script will recompile and rerun all tests
%   from the CMU Problem-Based Benchmark Suite.
% \end{closeitemize}


%%% Local Variables:
%%% mode: latex
%%% TeX-master: "tapir"
%%% End:
