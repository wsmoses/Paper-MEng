\newcommand{\arch}{\texttt{x86\_64}\xspace}
\newcommand{\AWSinst}{\texttt{c4.8xlarge}\xspace}

%\appendix

\chapter{Artifact description}

This guide describes how to set up \tapir/LLVM and how to download and
run our suite of application benchmarks.  In particular, this guide
focuses on setting up and running three software components:
\begin{closeitemize}
\item the \tapir/LLVM compiler,
\item the PClang front end to \tapir/LLVM, and
% \item The Intel Cilk Plus runtime system, and
\item the suite of \fillintheblank{20} Cilk application benchmarks
  described in \figref{bench}.
\end{closeitemize}
I provide instructions to download and build \tapir/LLVM and PClang.
I also provide instructions to download the application benchmark
suite and run the \tapir/LLVM compiler on that suite.

% This guide will lead you through the necessary steps to reproduce the
% experimental results given in \figref{benchnum}.  The figure compares
% the serial and parallel performance of 20 application benchmarks when
% compiled using either Tapir/LLVM or the reference compilation pipeline
% illustrated in \figref{pipeline}.  To reproduce these results, this
% evaluation makes use of four software components:
% \begin{closeitemize}
% \item The \tapir/LLVM compiler, which is the subject of the
%   evaluation,
% \item The PClang front end to \tapir/LLVM,
% % \item The Intel Cilk Plus runtime system, and
% \item The suite of 20 Cilk application benchmarks described in
%   \figref{bench}.
% \end{closeitemize}
% This guide details how to obtain these four software components, set
% them up, and run them in order to evaluate Tapir/LLVM's correctness
% and efficacy.

We have built and tested \tapir/LLVM, PClang, and the test suite on an
\arch shared-memory multicore machine running Linux.  We provide
instructions for obtaining \tapir/LLVM and PClang from my GitHub
repositories and setting up the compiler on such a machine.  Due to
the complexity of the LLVM compiler on which Tapir/LLVM is based,
building \tapir/LLVM requires significant computational resources:
approximately \fillintheblank{\SI{50}{\gibi\byte}} of disk,
\fillintheblank{\SI{12}{\gibi\byte}} of RAM, and anywhere from a few
minutes to a couple of hours, depending on the machine.  We also
provide instructions for obtaining a copy of our test suite from a
tarball.

\section{Building \tapir/LLVM from source}\label{sec:install}

This section describes how to download the source code for \tapir/LLVM
and PClang from GitHub and build them.  These instructions assume you
are building \tapir/LLVM on an \arch system running Linux.

\paragraph{System requirements.}  Building \tapir/LLVM and PClang
involves building the LLVM and Clang systems that they extend.
Because of the size of the underlying LLVM and Clang codebases, you
need a relatively powerful machine in order to build the compiler in a
timely fashion.  Approximately \SI{50}{\gibi\byte} of disk space and
\SI{12}{\gibi\byte} of memory are needed to compile LLVM and Clang.  A
fresh build of LLVM and Clang can take substantial time to complete,
e.g., approximately an hour on one processor of an AWS \AWSinst
instance.  The build script will attempt to use parallel processors to
speed up compilation.  See \url{http://llvm.org/docs/CMake.html} for
more information on building LLVM and Clang.

\begin{enumerate}
\item Install the requisite software to build \tapir/LLVM and PClang,
  namely, \code{cmake}, \code{gcc}, and \code{git}.

\item Download the sources of Tapir/LLVM and PClang from
  GitHub:
\begin{verbatim}
$ git clone --recursive https://github.com/wsmoses/Tapir-Meta.git
\end{verbatim}
  The source is approximately \fillintheblank{\SI{800}{\mebi\byte}} in
  size.

\item Compile \tapir/LLVM and PClang:
\begin{verbatim}
$ cd Tapir-Meta/
$ bash ./build.sh
\end{verbatim}
  This script will build \tapir/LLVM and PClang and store the compiled
  binaries in \code{Tapir-Meta/tapir/build}.  If the build succeeds, the
  final line of output will be \texttt{Installation successful}.

\item Set up your environment variables to use \tapir/LLVM and PClang:
\begin{verbatim}
$ source ./setup-env.sh
\end{verbatim}
  This script will add the \code{Tapir-Meta/tapir/build/bin/}
  subdirectory to your path, so that the \code{clang} command will
  refer to \tapir/LLVM and PClang.
\end{enumerate}


%%%%%%%%%%%%%%%%%%%%%%%%%%%%%%%%%%%%%%%%%%%%%%%%%%%%%%%%%%%%%%%%%%%%%
\section{Running the benchmark suite}\label{sec:experiment}

This section describes how you can download the application benchmark
suite described in \figref{bench} and test \tapir/LLVM on these
benchmarks.
\begin{enumerate}
\item Install the requisite software to download and run the tests,
  namely, \code{bc}, \code{libcilkrts}, \code{numactl}, \code{python},
  \code{taskset}, and \code{wget}.

\item Download the tarball containing the application benchmark suite
  and unpack it:
\begin{verbatim}
$ wget http://tinyurl.com/TapirLLVMTesting -O testing.tar
$ tar -xvf testing.tar
\end{verbatim}
  This tarball is approximately \fillintheblank{\SI{12}{\gibi\byte}}
  in size.  Unpacking the tarball creates the \code{testing/}
  subdirectory of the current working directory that contains the
  application benchmark suite.

\item Run the test script:
\begin{verbatim}
$ cd testing
$ ./test.sh
\end{verbatim}
  The test script takes approximately \fillintheblank{7} hours to run.
  The script compiles each benchmark in the test suite twice using
  \tapir/LLVM: once as a parallel program, and once as the program's
  serial elision.  All compilations use optimization level \code{-O3}.
  The test script runs each compiled executable \fillintheblank{$10$}
  times using $1$ worker thread and \fillintheblank{$10$} times using
  \fillintheblank{$18$} worker threads.
\end{enumerate}


\section{Evaluation and expected result}\label{sec:evaluate}

% {\em Obligatory}

Once the test script finishes running, the results can be summarized
into a table similar to \figref{benchnum} as follows:
\begin{verbatim}
$ ./results.sh > results.csv
\end{verbatim}
This command will produce \code{results.csv}, a table of tab-separated
values that contains the minimum running time from each set of
\fillintheblank{$10$} runs of a particular executable on a particular
worker count.  The table also contains derived work-efficiency and
parallel speedup values for each benchmark program.

Because these results are performance measurements, they are likely to
vary from run to run and from system to system.  Moreover, the Tapir team is
continuing to develop the \tapir/\mbox{LLVM} compiler and PClang,
meaning that your results will not precisely match those in
\figref{benchnum}.

You can write your own programs and compile them using PClang and
Tapir/LLVM\@.  The PClang front end is \emph{not} a fully featured
Cilk front end, however.  For more information on the source language
parsed by PClang, please see
\fillintheblank{\texttt{testing/PClang-README.txt}}.

