\chapput{aux}{Auxiliary software}

This chapter describes auxiliary software that the Tapir team developed to
exercise and test \tapir/LLVM\@.  Although our research focuses on the
middle end of the compiler, we implemented a front end for Cilk Plus.
In addition, we developed compiler instrumentation that allows the
compiler to interface to a race detector to verify the correctness of
the \tapir/LLVM implementation.

% and a pass to lower \tapir constructs to Intel Cilk Plus runtime calls

To create the front end, the Tapir team created a modification of the Clang front
end called PClang, which translates Cilk Plus codes to \tapir.  We
also created a version of Clang that can handle some OpenMP codes.
PClang handles most of the fork-join control constructs specified by
the Cilk Plus programming model, and specifically, enough to run all
the benchmarks described in \chapref{eval}.

We augmented \tapir/LLVM in two ways to test the correctness of the
implementation.  First, we modified LLVM's internal verification pass
to check that \tapir's invariants are also maintained.  Second, we
added an instrumentation pass to \tapir/LLVM to allow parallel
executables to be tested for determinacy races using a provably good
determinacy race detector.  This race detector, based on the SP-bags
algorithm \cite{FengLe99}, is guaranteed to find a determinacy race if
an only if one exists in the program execution.  The verification pass
and race detector helped us locate and fix bugs in \tapir/LLVM, both
within our code and within the underlying LLVM codebase.  \tapir/LLVM
now passes all tests in LLVM's regression test suites and correctly
compiles our own suite of parallel test programs.

% To test the correctness of \tapir/LLVM, we implemented two tools to
% check that \tapir/LLVM functions as expected.  First, we modified
% LLVM's internal verification pass to check that \tapir's invariants
% are also maintained.  Second, we augmented \tapir/LLVM to enable
% compiled executables to be tested for determinacy races, using detector
% based on the SP

% Second, we augmented the lowering pass to
% optionally instrument \detach, \reattach, and \sync instructions, as
% well as function entries and exits.  This instrumentation allowed us
% use a race detector based on the SP-bags algorithm \cite{FengLe99},
% which guarantees to identify a determinacy race if one exists.  

The instrumentation pass has proved useful for supporting other
dynamic-analysis tools based on \tapir/LLVM.  Genghis Chau of MIT
adapted the Cilkprof scalability profiler \cite{SchardlKuLe15} to use
\tapir/LLVM and this instrumentation in order to build an integrated
development environment with always-on race detection and scalability
profiling facilities.

% This same instrumentation also allowed us to efficiently examine the
% scalability of \tapir programs, using the Cilkprof scalability
% profiler \cite{SchardlKuLe15}, to ensure that optimizations do
% not % serialize code or otherwise
% inhibit parallel scalability.  

% Using these tools, we tested the correctness of \tapir/LLVM when run
% on the benchmarks from \secref{eval} and on many other codes.  The
% tools found bugs in our initial implementation of \tapir/LLVM as well
% as bugs within LLVM itself within the CodeGen phase.  After these bugs
% were patched, all the benchmarks behaved as expected.  We also ran
% \tapir/LLVM against LLVM's internal regression suite and verified that
% \tapir did not produce any additional failures.

%%% Local Variables:
%%% mode: latex
%%% TeX-master: "tapir"
%%% End:

%  LocalWords:  Cilkprof profiler CodeGen
%  LocalWords:  LLVM multithreading multicore OpenMP Cilk GCC Cilk's
%  LocalWords:  pseudocode metadata LLVM's bitcode codebase CFG's CFG
%  LocalWords:  subexpression interleavings subcomputations runtime
%  LocalWords:  vertices subcomputation OpenMP's inlining quicksort
%  LocalWords:  CSE LICM TRE inlined Hoare's
